\documentclass[11pt]{article}
\usepackage{url}
\usepackage{amsmath,amsfonts,amssymb,amsthm}
\usepackage{bbm,bm}
\usepackage{upgreek}
\usepackage{stmaryrd}
\usepackage{color}
\usepackage{tensors}

\def\SSCRSIZE{\scriptstyle}
\def\SSSCRSIZE{\scriptscriptstyle}

% * * * * * * * * * * * * * * * * * * * * * * * * * * * * * * * * * * * * * * * %
% * * * * * * * * * * * * *  M A T H  M A C R O S   * * * * * * * * * * * * * * %
% * * * * * * * * * * * * * * * * * * * * * * * * * * * * * * * * * * * * * * * %
% -------------------------------------------------- functional and number spaces
\newcommand{\CC}[1]{{\mathcal{C}^{#1}}}
\newcommand{\CCs}[2]{{\mathcal{C}^{#1}_{#2}}}
\newcommand{\Lsp}[1]{{\mathrm{L}^{\scriptstyle#1}}}
\newcommand{\Lo}{\Lsp{1}}
\newcommand{\Lt}{\Lsp{2}}
\newcommand{\Li}{\Lsp{\infty}}
\newcommand{\R}[1]{\mathbb{R}^{#1}}
\newcommand{\Rn}{\R{9}}
\newcommand{\Rs}{\R{6}}
\newcommand{\Rt}{\R{3}}

% ----------------------------------------------- differential calculus *sand/tim
\newcommand{\mathpp}{\partial}
\newcommand{\mathdd}{d}
\newcommand{\mathoprod}{} %{\otimes}

\newcommand{\Curl}{\operatorname{curl}}
\newcommand{\DBar}{\mathchar'26\mkern-12mu {\rm d}}
\newcommand{\Der}[2]{\frac{\mathdd{#1}}{\mathdd{#2}}}
\newcommand{\DerN}[3]{\frac{\mathdd^{#1}{#2}}{\mathdd{#3}^{#1}}}
\newcommand{\Div}{\operatorname{div}}
\newcommand{\DPolar}[1]{\mathring{#1}}
\newcommand{\FourDer}[2]{\frac{\mathdd^4 #1}{\mathdd{#2}^4}}
\newcommand{\FiveDer}[2]{\frac{\mathdd^5 #1}{\mathdd{#2}^5}}
\newcommand{\Grad}{\operatorname{grad}}
\newcommand{\InexDer}[2]{\frac{\dbar #1}{\mathdd{#2}}} % inexact deriv
\newcommand{\LeftDel}{{\raise 1.45ex\hbox{$\scriptstyle\bm\leftharpoonup$}}%
{\kern -.85em\hbox{$\nabla$}}}
\newcommand{\LeftDer}[2]{\dfrac{\mathdd{}}{\mathdd{#2}}{#1}}
\newcommand{\LeftDerN}[3]{\frac{\mathdd^{#1}}{\mathdd{#3}^{#1}}{#2}}
\newcommand{\LPDer}[2]{{\frac{\mathpp}{\mathpp #2}#1}}
\newcommand{\LPDerN}[3]{\frac{\mathpp^{#1}}{\mathpp{#3}^{#1}}{#2}}
\newcommand{\LPXDer}[3]{\frac{\mathpp^2}{\mathpp{#2}\mathpp{#3}}{#1}}
\newcommand{\MatDer}[1]{\frac{\mrm{D} #1}{\mrm{D}t}}
\newcommand{\Nc}[1]{\nabla\times{#1}}
\newcommand{\Nd}[1]{\nabla\cdot{#1}}
\newcommand{\Ng}[1]{\nabla\,{#1}}
\newcommand{\PDer}[2]{\ensuremath{\frac{\mathpp #1}{\mathpp #2}}}
\newcommand{\PDerC}[3]{\left(\frac{\mathpp #1}{\mathpp #2}\right)_{\!\negthickspace#3}}
\newcommand{\PDerN}[3]{\frac{\mathpp^{#1}{#2}}{\mathpp{#3}^{#1}}}
\newcommand{\PXDer}[3]{\frac{\mathpp^2{#1}}{\mathpp{#2}\mathpp{#3}}}
\newcommand{\RightDel}{{\raise 1.45ex\hbox{$\scriptstyle\bm\rightharpoonup$}}%
{\kern -.85em\hbox{$\nabla$}}}
\newcommand{\SDer}[2]{\ensuremath{ \frac{\mathdd{}^2#1}{\mathdd{#2}^2}}}
\newcommand{\SPDer}[3]{% second order partial
  \ifthenelse{\equal{#2}{#3}}{%
    \frac{\mathpp^2 #1}{\mathpp{#2}^2}}{%
    \frac{\mathpp^2 #1}{\mathpp{#2}\mathpp{#3}}}}
\newcommand{\SPDerMixed}[3]{\frac{\mathpp #1}{{\mathpp #2}{\mathpp #3}}}
\newcommand{\SPDerC}[4]{%mixed second order partial with constant
  \ifthenelse{\equal{#2}{#3}}{%
    \left(\frac{\mathpp^2 #1}{\mathpp{#2}^2}\right)_{#4}}{%
    \left(\frac{\mathpp^2 #1}{\mathpp{#2}\mathpp{#3}}\right)_{#4}}}
\newcommand{\SPDerTens}[3]{\frac{\mathpp^2 #1}{{\mathpp #2}{\mathoprod}{\mathpp #3}}}
\newcommand{\TDer}[2]{\frac{\mathdd^3 #1}{\mathdd{#2}^3}}
\newcommand{\TPDer}[4]{\ensuremath{\frac{\mathpp^3 #1}{\mathpp #2 \mathpp #3 \mathpp #4}}}
\newcommand{\DefInt}[4]{\int_{#1}^{#2}{#3}\,d#4}
% ------------------------------ differential volumes, lengths, areas, time, etc.
\newcommand{\da}{\,\mathdd{a}}
\newcommand{\dA}{\,\mathdd{A}}
\newcommand{\dt}{\,\mathdd{t}}
\newcommand{\Dt}{\Delta{t}}
\newcommand{\dV}{\,\mathdd{V}}
\newcommand{\dx}{\,\mathdd{x}}

% --------------------------------------------------------------- vector calculus
\newcommand{\Norm}[1]{\left\lVert{#1}\right\rVert}
\newcommand{\NormOne}[1]{\norm{#1}_1}
\newcommand{\NormTwo}[1]{\norm{#1}_2}
\newcommand{\NormSup}[1]{\norm{#1}_\infty}
\newcommand{\NormGen}[2]{\norm{#2}_{#1}}
\newcommand{\NormLo}[1]{\norm{#1}_\Lo}
\newcommand{\NormLt}[1]{\norm{#1}_\Lt}
\newcommand{\NormLi}[1]{\norm{#1}_\Li}
\newcommand{\NormL}[2]{\norm{#2}_\Lsp{#1}}
\newcommand{\SemiNorm}[2]{\left\lvert{#2}\right\rvert_{#1}}
\newcommand{\IndNorm}[1]{\left\lVert\!\left\lvert{#1}\right\rVert\!\right\lvert}
\newcommand{\InnProd}[2]{\left({#1}|{#2}\right)}
\newcommand{\DualPair}[2]{\left\langle{#1},{#2}\right\rangle}
\newcommand{\MixProd}[3]{\left[{#1},{#2},{#3}\right]}

% ---------------------------------------------------- asymptotic notations *sand
\newcommand{\SmallO}[1]{\operatorname{o}\left({#1}\right)}
\newcommand{\BigO}[1]{\operatorname{O}\left({#1}\right)}
\newcommand{\ASeq}[2]{\operatornamewithlimits{\sim}_{{#1}\rightarrow{#2}}}

% ---------------------------------------------- Misc. math operators and symbols
\newcommand{\mathopsupscrtext}[1]{^{\SSCRSIZE\mathrm{#1}}}
\newcommand{\mathopsubscrtext}[1]{_{\SSCRSIZE\mathrm{#1}}}
\newcommand{\mathopsupscrmath}[1]{^{\SSCRSIZE{#1}}}
\newcommand{\mathopsubscrmath}[1]{_{\SSCRSIZE{#1}}}
\newcommand{\Abs}[1]{\left\lvert#1\right\rvert}
\newcommand{\AbsOp}{\operatorname{abs}}
\newcommand{\Boundary}{\partial\Domain}
\newcommand{\Domain}{\Omega}
\newcommand{\E}[1]{\mathrm{e}\mathopsupscrmath{#1}}
\newcommand{\OrderOf}[1]{\mathcal{O}\left(#1\right)}
\newcommand{\x}{\TENSTYPESETTYPE{x}}

% * * * * * * * * * * * * * * * * * * * * * * * * * * * * * * * * * * * * * * * %
% * * * * * * * * * *  M E C H A N I C S  M A C R O S   * * * * * * * * * * * * %
% * * * * * * * * * * * * * * * * * * * * * * * * * * * * * * * * * * * * * * * %
% -------------------------------------------------------------- mechanics macros
% --- internally used
\newcommand{\TENSTYPESETTYPE}[1]{\boldsymbol{#1}}
\newcommand{\mechopsupscrtext}[1]{^{\SSCRSIZE\text{#1}}}
\newcommand{\mechopsubscrtext}[1]{_{\SSCRSIZE\text{#1}}}
\newcommand{\mechopsupscrmath}[1]{^{\SSCRSIZE{#1}}}
\newcommand{\mechopsubscrmath}[1]{_{\SSCRSIZE{#1}}}
\newcommand{\mechopgenquant}[1]{{\overline{#1}}{\hspace{.5pt}}}

% ------------------------------------- symbols, change here to change throughout
\newcommand{\BackStressSym}{\alpha}
\newcommand{\EulerStrainSym}{e}
\newcommand{\GreenLagrangeStrainSym}{E}
\newcommand{\RightCauchyGreenStretchSym}{B}
\newcommand{\LogStrainSym}{\varepsilon}
\newcommand{\DevLogStrainSym}{\gamma}
\newcommand{\GenStrainSym}{\mechopgenquant{\epsilon}}
\newcommand{\DevGenStrainSym}{\mechopgenquant{\gamma}}
\newcommand{\GenHillStrainSym}{\mechopgenquant{h}}
\newcommand{\StrainSym}{\epsilon}
\newcommand{\dStrainSym}{\INCRTYP{\epsilon}}
\newcommand{\DevStrainSym}{\gamma}
\newcommand{\SymVelGradSym}{d}
\newcommand{\VelGradSym}{L}
\newcommand{\DefGradSym}{F}
\newcommand{\CauchyStressSym}{\sigma}
\newcommand{\DevCauchyStressSym}{S}
\newcommand{\KirStressSym}{\tau}
\newcommand{\DevKirStressSym}{\mathcal{S}}
\newcommand{\StressSym}{\sigma}
\newcommand{\DevStressSym}{S}
\newcommand{\GenStressSym}{\mechopgenquant{\StressSym}}
\newcommand{\GenShearStrengthSym}{\mechopgenquant{\tau}}
\newcommand{\ShiftedStressSym}{\xi}
\newcommand{\MechInvarSymO}{J}
\newcommand{\MechInvarSymT}{J}


% ----------------------------------------------------------------------------- %
\def\INCRTYP{\Rate}
\newcommand{\Incr}[1]{d{#1}}
\newcommand{\Rate}[1]{\dot{#1}}
\newcommand{\HypoQuant}[1]{\widetilde{#1}}              % hypoelastic quantity
\newcommand{\HypoRate}[1]{\INCRTYP{\HypoQuant{#1}}}
\newcommand{\Trial}[1]{#1\mechopsupscrtext{trial}}
\newcommand{\Over}[1]{#1\mechopsupscrtext{over}}
\newcommand{\Test}[1]{#1\mechopsupscrtext{test}}
\newcommand{\Quasistatic}[1]{#1\mechopsupscrtext{qs}}
\newcommand{\ElastPart}[1]{#1\mechopsupscrtext{e}}
\newcommand{\PlastPart}[1]{#1\mechopsupscrtext{p}}
\newcommand{\DevPart}[1]{#1^{\prime}}
\newcommand{\IsoPart}[1]{#1\mechopsupscrtext{iso}}
\newcommand{\QuasiStatic}[1]{#1\mechopsubscrtext{qs}}
\newcommand{\Jump}[1]{\llbracket#1\rrbracket}

% --------------------------------------------------------------- stress measures
\newcommand{\BackStress}{\TENSTYPESETTYPE{\BackStressSym}}
\newcommand{\BackStressComp}[1]{\BackStressSym\mechopsubscrmath{#1}}
\newcommand{\dBackStress}{\INCRTYP{\BackStress}}
\newcommand{\ShiftedStress}{\TENSTYPESETTYPE{\ShiftedStressSym}}
\newcommand{\ScalarStress}{\StressSym}
\newcommand{\dScalarStress}{\INCRTYP{\ScalarStress}}

% ---------------------------------------------------------------- general stress
\newcommand{\DevGenStress}{\DevPart{\GenStress}}
\newcommand{\dGenStress}{\INCRTYP{\GenStress}}
\newcommand{\dGenStressSym}{\INCRTYP{\GenStressSym}}
\newcommand{\dHypoGenStress}{\HypoINCRTYP{\GenStress}}
\newcommand{\dTrialGenStress}{\Trial{\INCRTYP{\GenStress}}}
\newcommand{\GenStress}{\TENSTYPESETTYPE{\GenStressSym}}
\newcommand{\OverGenStress}{{\Over{\GenStress}}} %overstress
\newcommand{\QSGenStress}{{\Quasistatic{\GenStress}}} %overstress
\newcommand{\TrialGenStress}{\Trial{\GenStress}}
\newcommand{\GenShearStrength}{\GenShearStrengthSym\mechopsubscrtext{y}}
\newcommand{\GenZ}{\GenStressSym\mechopsubscrtext{z}}
\newcommand{\dGenZ}{\INCRTYP{\GenStressSym}\mechopsubscrtext{z}}
\newcommand{\dTrialGenZ}{\Trial{\dGenZ}}
\newcommand{\GenR}{\GenStressSym\mechopsubscrtext{r}}
\newcommand{\dGenR}{\INCRTYP{\GenStressSym}\mechopsubscrtext{r}}
\newcommand{\dTrialGenR}{\Trial{\dGenR}}
\newcommand{\GenRy}{\GenStressSym\mechopsubscrtext{ry}}

% ------------------------------------------------------------------------ stress
\newcommand{\dStress}{\INCRTYP{\Stress}}
\newcommand{\dStressInd}[1]{\INCRTYP{\StressSym}\mechopsubscrmath{#1}}
\newcommand{\Stress}{\TENSTYPESETTYPE{\StressSym}}
\newcommand{\StressInd}[1]{\StressSym\mechopsubscrmath{#1}}
\newcommand{\DevStress}{\TENSTYPESETTYPE{\DevStressSym}}
\newcommand{\dDevStress}{\INCRTYP{\DevStress}}
\newcommand{\TrialStress}{\Trial{\Stress}}

% ---------------------------------------------------------------------- pressure
\newcommand{\Pressure}{p}
\newcommand{\GenPressure}{\mechopgenquant{\Pressure}}
\newcommand{\dGenPressure}{\INCRTYP{\GenPressure}}
\newcommand{\dPressure}{\INCRTYP{\Pressure}}
\newcommand{\InitPressure}{\Pressure\mechopsubscrmath{0}}

% -------------------------------------------------------------- Kirchhoff stress
\newcommand{\dKirStress}{\INCRTYP{\KirStress}}
\newcommand{\dKirStressComp}[1]{\INCRTYP{\KirStressSym}\mechopsubscrmath{#1}}
\newcommand{\KirStressComp}[1]{\KirStressSym\mechopsubscrmath{#1}}
\newcommand{\KirStress}{\TENSTYPESETTYPE{\KirStressSym}}
\newcommand{\KirStressInd}[1]{\KirStressSym\mechopsubscrmath{#1}}
\newcommand{\DevKirStress}{\TENSTYPESETTYPE{\DevKirStressSym}}
\newcommand{\dDevKirStress}{\INCRTYP{\DevKirStress}}
\newcommand{\TrialKirStress}{\Trial{\KirStress}}
\newcommand{\dTrialKirStress}{\Trial{\dKirStress}}

% ----------------------------------------------------------------- Cauchy stress
\newcommand{\dCauchyStress}{\INCRTYP{\CauchyStress}}
\newcommand{\dCauchyStressInd}[1]{\INCRTYP{\CauchyStressSym}\mechopsubscrmath{#1}}
\newcommand{\CauchyStress}{\TENSTYPESETTYPE{\CauchyStressSym}}
\newcommand{\CauchyStressInd}[1]{\CauchyStressSym\mechopsubscrmath{#1}}
\newcommand{\DevCauchyStress}{\TENSTYPESETTYPE{\DevCauchyStressSym}}
\newcommand{\dDevCauchyStress}{\INCRTYP{\DevCauchyStress}}
\newcommand{\TrialCauchyStress}{\Trial{\CauchyStress}}

% ------------------------------------------------------------- Stress invariants
\newcommand{\IOne}[1]{I\mechopsubscrmath{1}\mechopsupscrmath{#1}}
\newcommand{\TrialIOne}[1]{\Trial{\IOne{#1}}}
\newcommand{\JTwo}[1]{J\mechopsubscrmath{2}\mechopsupscrmath{#1}}
\newcommand{\TrialJTwo}[1]{\Trial{\JTwo{#1}}}
\newcommand{\LodeAng}[1]{\theta\mechopsupscrmath{#1}}

% ---------------------------------------------------------- deformation measures
% ---------------------------------------------------------------- General Strain
\newcommand{\dGenStrain}{\INCRTYP{\GenStrain}}
\newcommand{\dGenStrainSym}{\INCRTYP{\GenStrainSym}}
\newcommand{\dVolGenStrain}{\dGenStrain\mechopsubscrtext{v}}
\newcommand{\dGenStrainInd}[1]{\INCRTYP{\GenStrainSym}\mechopsubscrmath{#1}}
\newcommand{\GenStrain}{\TENSTYPESETTYPE{\GenStrainSym}}
\newcommand{\GenStrainInd}[1]{\GenStrainSym\mechopsubscrmath{#1}}
\newcommand{\DevGenStrain}{\TENSTYPESETTYPE{\DevGenStrainSym}}
\newcommand{\DevElastGenStrain}{\ElastPart{\DevGenStrain}}
\newcommand{\DevPlastGenStrain}{\PlastPart{\DevGenStrain}}
\newcommand{\dDevGenStrain}{\INCRTYP{\DevGenStrain}}
\newcommand{\dElastGenStrain}{\ElastPart{\dGenStrain}}
\newcommand{\dDevElastGenStrain}{\ElastPart{\dDevGenStrain}}
\newcommand{\dPlastGenStrain}{\PlastPart{\dGenStrain}}
\newcommand{\ElastGenStrain}{\ElastPart{\GenStrain}}
\newcommand{\ElastDevGenStrain}{\ElastPart{\DevGenStrain}}
\newcommand{\dElastDevGenStrain}{\ElastPart{\INCRTYP{\DevGenStrain}}}
\newcommand{\PlastGenStrain}{\PlastPart{\GenStrain}}
\newcommand{\dGenVolStrain}{\INCRTYP{\GenStrainSym}\mechopsubscrtext{v}}
\newcommand{\dElastGenVolStrain}{\ElastPart{\dGenVolStrain}}
\newcommand{\GenHillStrain}{\TENSTYPESETTYPE{\GenHillStrainSym}}

% ------------------------------------------------------------------------ Strain
\newcommand{\dStrain}{\INCRTYP{\Strain}}
\newcommand{\dStrainInd}[1]{\INCRTYP{\StrainSym}\mechopsubscrmath{#1}}
\newcommand{\ScalarStrain}{\StrainSym}
\newcommand{\dScalarStrain}{\INCRTYP{\ScalarStrain}}
\newcommand{\Strain}{\TENSTYPESETTYPE{\StrainSym}}
\newcommand{\StrainInd}[1]{\StrainSym\mechopsubscrmath{#1}}
\newcommand{\DevStrain}{\TENSTYPESETTYPE{\DevStrainSym}}
\newcommand{\dDevStrain}{\INCRTYP{\DevStrain}}
\newcommand{\dElastStrain}{\ElastPart{\dStrain}}
\newcommand{\dPlastStrain}{\PlastPart{\dStrain}}
\newcommand{\PlastStrain}{\PlastPart{\Strain}}
\newcommand{\ElastStrain}{\ElastPart{\Strain}}
\newcommand{\dVolStrain}{\INCRTYP{\StrainSym}\mechopsubscrtext{v}}
\newcommand{\VolStrain}{\StrainSym\mechopsubscrtext{v}}

% -------------------------------------------------------------------- Log Strain
\newcommand{\dLogStrain}{\INCRTYP{\LogStrain}}
\newcommand{\dLogStrainSym}{\INCRTYP{\LogStrainSym}}
\newcommand{\dVolLogStrain}{\dLogStrain\mechopsubscrtext{v}}
\newcommand{\dLogStrainInd}[1]{\INCRTYP{\LogStrainSym}\mechopsubscrmath{#1}}
\newcommand{\LogStrain}{\TENSTYPESETTYPE{\LogStrainSym}}
\newcommand{\LogStrainInd}[1]{\LogStrainSym\mechopsubscrmath{#1}}
\newcommand{\DevLogStrain}{\TENSTYPESETTYPE{\DevLogStrainSym}}
\newcommand{\dDevLogStrain}{\INCRTYP{\DevLogStrain}}
\newcommand{\dPlastDevLogStrain}{\PlastPart{\dDevLogStrain}}
\newcommand{\dElastDevLogStrain}{\ElastPart{\dDevLogStrain}}
\newcommand{\dElastLogStrain}{\ElastPart{\dLogStrain}}
\newcommand{\dPlastLogStrain}{\PlastPart{\dLogStrain}}
\newcommand{\dPlastVolLogStrain}{\PlastPart{\dLogStrain}\mechopsubscrtext{v}}
\newcommand{\ElastLogStrain}{\ElastPart{\LogStrain}}
\newcommand{\DevElastLogStrain}{\ElastPart{\DevLogStrain}}
\newcommand{\PlastLogStrain}{\PlastPart{\LogStrain}}
\newcommand{\VolLogStrain}{\LogStrain\mechopsubscrtext{v}}

% ------------------------------------------------------------- velocity gradient
\newcommand{\SymVelGrad}{{\TENSTYPESETTYPE{\SymVelGradSym}}}
\newcommand{\ElastSymVelGrad}{\ElastPart{\symvelgrad}}
\newcommand{\PlastSymVelGrad}{\PlastPart{\symvelgrad}}
\newcommand{\VelGrad}{{\TENSTYPESETTYPE{\VelGradSym}}}

% ---------------------------------------------------------- deformation gradient
\newcommand{\Mapping}{\varphi}
\newcommand{\DefGrad}{\TENSTYPESETTYPE{\DefGradSym}}
\newcommand{\ElastDefGrad}{\ElastPart{\DefGrad}}
\newcommand{\PlastDefGrad}{\PlastPart{\DefGrad}}
\newcommand{\Jacobian}{J}  % jacobian
\newcommand{\ElastJacobian}{\ElastPart{\Jacobian}}
\newcommand{\PlastJacobian}{\PlastPart{\Jacobian}}

% --------------------------------------------------------- Green-Lagrange strain
\newcommand{\GreenLagrangeStrain}{\TENSTYPESETTYPE{\GreenLagrangeStrainSym}}

% ---------------------------------------------------- Right Cauchy-Green stretch
\newcommand{\RightCauchyGreenStretch}{\TENSTYPESETTYPE{\RightCauchyGreenStretchSym}}

% ------------------------------------------------------------------ Euler strain
\newcommand{\EulerStrain}{\TENSTYPESETTYPE{\EulerStrainSym}}

% -------------------------------------------------------------- strain invariants
\newcommand{\dFMechInv}[1]{{\INCRTYP{\MechInvarSymO}\mechopsubscrmath{1}\mechopsupscrmath{#1}}}
\newcommand{\dSMechInv}[1]{{\INCRTYP{\MechInvarSymT}\mechopsubscrmath{2}\mechopsupscrmath{#1}}}
\newcommand{\dTMechInv}[1]{{\INCRTYP{\MechInvarSymT}\mechopsubscrmath{3}\mechopsupscrmath{#1}}}
\newcommand{\FMechInv}[1]{{\MechInvarSymO\mechopsubscrmath{1}\mechopsupscrmath{#1}}}
\newcommand{\SMechInv}[1]{{\MechInvarSymT\mechopsubscrmath{2}\mechopsupscrmath{#1}}}
\newcommand{\TMechInv}[1]{{\MechInvarSymT\mechopsubscrmath{3}\mechopsupscrmath{#1}}}
\newcommand{\MechInv}[2]{{\MechInvarSymT\mechopsubscrmath{#1}\mechopsupscrmath{#2}}}

% ------------------------------------------------------------------ elasticities
% ---------------------------------------------------------- fourth order tensors
\newcommand{\Compliance}{\mathbb{S}}
\newcommand{\IsoStiffness}{\Stiffness\mechopsubscrmath{\Tmpr}}
\newcommand{\IsenStiffness}{\Stiffness\mechopsubscrmath{\SpecEntropy}}
\newcommand{\Stiffness}{\mathbb{C}}
\newcommand{\PlastStiffness}{\mathbb{K}}

% ------------------------------------------------------------------------ moduli
\newcommand{\BMod}{\kappa}
\newcommand{\SMod}{\mu}
\newcommand{\EMod}{E}
\newcommand{\ConMod}{H}
\newcommand{\PoisRatio}{\nu}

% -------------------------------------------------------------------- mass stuff
\newcommand{\Density}{\rho}
\newcommand{\dDensity}{\INCRTYP{\Density}}
\newcommand{\InitDensity}{\Density\mechopsubscrmath{0}}
\newcommand{\Volume}{V}
\newcommand{\InitVolume}{\Volume\mechopsubscrmath{0}}

% ----------------------------------------------------------------- Thermal stuff
\newcommand{\DissVars}{\chi}
\newcommand{\dDissVars}{\INCRTYP{\DissVars}}
\newcommand{\Dissipation}{\mathcal{D}}
\newcommand{\QDissipation}{\mathcal{F}}    % dissipation
\newcommand{\HeatDissipation}{\mathcal{F}}
\newcommand{\SpecEnthalpy}{h}
\newcommand{\SpecEnrgy}{u}
\newcommand{\EnrgyProd}{r}
\newcommand{\TotEnrgy}{e}
\newcommand{\dSpecEnrgy}{\INCRTYP{\SpecEnrgy}}
\newcommand{\InitSpecEnrgy}{\SpecEnrgy\mechopsubscrmath{0}}
\newcommand{\SpecFEnrgy}{a}
\newcommand{\dSpecFEnrgy}{\INCRTYP{\Fenrgy}}
\newcommand{\dTmpr}{\INCRTYP{\Tmpr}}
\newcommand{\dTrialTmpr}{\Trial{\dTmpr}}
\newcommand{\Tmpr}{\theta}
\newcommand{\InitTmpr}{\Tmpr\mechopsubscrmath{0}}
\newcommand{\MeltTmpr}{\Tmpr\mechopsubscrtext{m}}
\newcommand{\dSpecEntropy}{\INCRTYP{\SpecEntropy}}
\newcommand{\SpecEntropy}{\eta}
\newcommand{\TotEntropy}{S}
\newcommand{\GrunPar}{{\gamma}}
\newcommand{\GrunTens}{{\TENSTYPESETTYPE{\Gamma}}}
\newcommand{\Heat}{Q}
\newcommand{\HeatFlux}{\TENSTYPESETTYPE{q}}
\newcommand{\Power}[1]{\mathcal{P}\mechopsubscrtext{#1}}
\newcommand{\SpHtV}{c\mechopsubscrtext{v}}

% -------------------------------------------------------------- Plasticity stuff
\newcommand{\YldFcn}[1]{\ifthenelse{\equal{#1}{}}{f}{f\left(#1\right)}}
\newcommand{\LSFcn}[1]{\ifthenelse{\equal{#1}{}}{L}{L\left(#1\right)}}
\newcommand{\GTFcn}[1]{\ifthenelse{\equal{#1}{}}{\boldsymbol{g}}%
   {\boldsymbol{g}\left(#1\right)}}
\newcommand{\CoupleTens}{\TENSTYPESETTYPE{\Uplambda}}
\newcommand{\EnsHardModulus}{H}
\newcommand{\FlowDir}{\TENSTYPESETTYPE{m}}
\newcommand{\GenFlowDir}{\TENSTYPESETTYPE{\mechopgenquant{m}}}
\newcommand{\FlowPotential}[1]{\ifthenelse{\equal{#1}{}}{\varphi}{\varphi\left(#1\right)}}
\newcommand{\YldNormal}{{\TENSTYPESETTYPE{n}}}
\newcommand{\Normal}{{\TENSTYPESETTYPE{n}}}
\newcommand{\GenNormal}{{\mechopgenquant{\Normal}}}
\newcommand{\GenYldNormal}{{\TENSTYPESETTYPE{\mechopgenquant{n}}}}
\newcommand{\GenMpsr}{\INCRTYP{\mechopgenquant{\lambda}}}
\newcommand{\GenMpsi}{\dd{\mechopgenquant{\lambda}}}
\newcommand{\GenMps}{\mechopgenquant{\Mps}}
\newcommand{\Mpsr}{\INCRTYP{\lambda}}
\newcommand{\Mps}{\Lambda}
\newcommand{\ReturnDir}{{\TENSTYPESETTYPE{p}}}
\newcommand{\GenReturnDirZ}{\mechopgenquant{p}\mechopsubscrtext{z}}
\newcommand{\GenReturnDirR}{\mechopgenquant{p}\mechopsubscrtext{r}}
\newcommand{\GenReturnDir}{{\TENSTYPESETTYPE{\mechopgenquant{p}}}}
\newcommand{\TmprProj}{P\mechopsubscrmath{\Tmpr}}
\newcommand{\TmprModulus}{\Theta}               % temperature modulus
\newcommand{\EntropyModulus}{\Xi}               % temperature modulus
\newcommand{\TQC}{\chi}                     % Taylor Quinney coefficient
\newcommand{\dIsv}[1]{\INCRTYP{\Isv{#1}}}
\newcommand{\Isv}[1]{q\mechopsubscrmath{#1}}
\newcommand{\IsvModulus}[1]{h\mechopsubscrmath{#1}}
\newcommand{\IsvMod}{G}
\newcommand{\EnIsvModulus}[1]{\psi\mechopsubscrmath{#1}}
\newcommand{\IntVars}{\xi}
\newcommand{\IntVarsModulus}{\Xi}
\newcommand{\dIntVars}{\INCRTYP{\IntVars}}
\newcommand{\RelaxTime}{\tau}               % relaxation time
\newcommand{\EqPS}{{\PlastPart{\gamma\mechopsubscrtext{eq}}}}
\newcommand{\EqPV}{{\PlastPart{\varepsilon\mechopsubscrtext{v}}}}
\newcommand{\GenEqPV}{\PlastPart{\mechopgenquant{\epsilon}}\mechopsubscrtext{v}}
\newcommand{\dEqPS}{\INCRTYP{\EqPS}}
\newcommand{\PlastWork}{\Pi\mechopsubscrtext{p}}

% * * * * * * * * * * * * * * * * * * * * * * * * * * * * * * * * * * * * * * * %
% * * * * * * * * * *  K I N E M A T I C S  M A C R O S * * * * * * * * * * * * %
% * * * * * * * * * * * * * * * * * * * * * * * * * * * * * * * * * * * * * * * %
% ----------------------------------------------------------- kinematic variables
\newcommand{\Accel}{\TENSTYPESETTYPE{a}}
\newcommand{\BodyForce}{\TENSTYPESETTYPE{b}}
\newcommand{\SoundSpd}{{c}}
\newcommand{\Spd}{v}
\newcommand{\ShockSpd}{\Spd_s}
\newcommand{\Disp}{\TENSTYPESETTYPE{u}}
\newcommand{\Vlcty}{\TENSTYPESETTYPE{\Spd}}
\newcommand{\WaveSpeed}[1]{C\mechopsubscrmath{#1}}
\newcommand{\LongSpeed}{\WaveSpeed{L}}
\newcommand{\BulkSpeed}{\WaveSpeed{B}}
\newcommand{\ShearSpeed}{\WaveSpeed{S}}
\newcommand{\NormalVelocity}{\Spd_{\scriptscriptstyle\text{n}}}


% * * * * * * * * * * * * * * * * * * * * * * * * * * * * * * * * * * * * * * * %
% * * * * * * * * * * * * * * T E N S O R  M A C R O S  * * * * * * * * * * * * %
% * * * * * * * * * * * * * * * * * * * * * * * * * * * * * * * * * * * * * * * %
% ------------------------------------------------------------------ tensor types
\newcommand{\FOT}[1]{\ensuremath{\mathbb{#1}}}  % fourth order tensor
\newcommand{\UnderTens}[2]{%
   \ifthenelse{\equal{#2}{1}}{\ensuremath{\utw{\bm{#1}}}}{}
   \ifthenelse{\equal{#2}{2}}{\ensuremath{\utw{\utw{\bm{#1}}}}}{}
   \ifthenelse{\equal{#2}{3}}{\ensuremath{\utw{\utw{\utw{\bm{#1}}}}}}{}
   \ifthenelse{\equal{#2}{4}}{\ensuremath{\utw{\utw{\utw{\utw{\bm{#1}}}}}}}{}
}

% ----------------------------------------------------------- internal shortcuts
\newcommand{\tensopsupscrtext}[1]{^{\SSCRSIZE\mathrm{#1}}}
\newcommand{\tensopsubscrtext}[1]{_{\SSCRSIZE\mathrm{#1}}}
\newcommand{\tensopsupscrmath}[1]{^{\SSCRSIZE{#1}}}
\newcommand{\tensopsubscrmath}[1]{_{\SSCRSIZE{#1}}}
\newcommand{\tenssymdevpart}[1]{#1\tensopsupscrtext{sd}}
\newcommand{\tensskewpart}[1]{#1\tensopsupscrtext{skew}}
\newcommand{\tensisopart}[1]{#1\tensopsupscrtext{iso}}

% ------------------------------------------------------------- tensor operations
\newcommand{\Adj}[1]{#1\tensopsupscrtext{CT}} % adjugate
\newcommand{\ArbProd}{\odot}
\newcommand{\Cof}[1]{#1\tensopsupscrtext{C}}
\newcommand{\Cofi}[2]{\Cof{#1}\tensopsubscrmath{#2}}      % indicial cofactor
\newcommand{\Dev}{\operatorname{dev}}
\newcommand{\DevTens}[1]{{#1^{\prime}}}
%\newcommand{\DotProd}{{\raise .2ex\hbox{$\scriptscriptstyle{\bullet}{}$}}}
\newcommand{\DotProd}{{\boldsymbol{\cdot}}}
\newcommand{\RaylProd}{{\ast}}
%\newcommand{\DDotProd}{{{\raise .05ex\hbox{$\scriptscriptstyle{\bullet}$}}%
%             {\kern -.34em\raise .7ex\hbox{$\scriptscriptstyle{\bullet}$}}}}
\newcommand{\DDotProd}{{\boldsymbol{:}}}
\newcommand{\Inv}[1]{#1\tensopsupscrmath{-1}}
\newcommand{\Inverse}[1]{#1\tensopsupscrmath{-1}}
\newcommand{\InvTrans}[1]{#1\tensopsupscrtext{-T}}
\newcommand{\Iso}{\operatorname{iso}}
\newcommand{\OutProd}{{}} % {\otimes}
\newcommand{\TensInnProd}{{\raise .2ex\hbox{$\scriptscriptstyle{\bullet}{}$}}}
\newcommand{\GenInnProd}{{\raise .2ex\hbox{$\scriptscriptstyle{\bullet}{}$}}}
\newcommand{\QDotProd}{{\DDotProd{}}{\DDotProd{}}}
\newcommand{\SymOp}{\operatorname{sym}}
\newcommand{\TDotProd}{{{{\lower .2ex\hbox{$\scriptscriptstyle{\bullet}$}}%
           {\kern -.325em\raise .41ex\hbox{$\scriptscriptstyle{\bullet}$}}}%
           {\kern -.320em\raise  1.ex\hbox{$\scriptscriptstyle{\bullet}$}}}}
\newcommand{\Trans}[1]{\Transpose{#1}}
\newcommand{\Transpose}[1]{#1\tensopsupscrtext{T}}
\newcommand{\TraceOp}{\operatorname{tr}}
\newcommand{\Trace}[1]{\TraceOp#1}
\newcommand{\DetOp}{\operatorname{det}}
\newcommand{\Determinant}[1]{\DetOp#1}

% ----------------------------------------------------------------- Misc. tensors
\newcommand{\TensSymDevPart}[1]{#1\tensopsupscrtext{sd}}
\newcommand{\TensSkewPart}[1]{#1\tensopsupscrtext{skew}}
\newcommand{\TensIsoPart}[1]{#1\tensopsupscrtext{iso}}

% ----------------------------------------------------------- first order tensors
\newcommand{\aaa}{\TENSTYPESETTYPE{a}}
\newcommand{\bbb}{\TENSTYPESETTYPE{b}}
\newcommand{\ccc}{\TENSTYPESETTYPE{c}}

% ---------------------------------------------------------- second order tensors
\newcommand{\A}{\TENSTYPESETTYPE{A}}
\newcommand{\GenA}{\mechopgenquant{\A}}
\newcommand{\B}{\TENSTYPESETTYPE{B}}
\newcommand{\Cc}{\TENSTYPESETTYPE{C}}
\newcommand{\SpatPosition}{\TENSTYPESETTYPE{x}}
\newcommand{\RefPosition}{\TENSTYPESETTYPE{X}}
\newcommand{\BaseTens}[1]{\textbf{e}\tensopsubscrmath{#1}}
\newcommand{\FOBaseTens}[1]{\textbf{E}\tensopsubscrmath{#1}}
\newcommand{\Identity}{{\TENSTYPESETTYPE{\delta}}}
\newcommand{\IndIdentity}[1]{\delta\tensopsubscrmath{#1}}
\newcommand{\ZeroTens}{\TENSTYPESETTYPE{0}}
\newcommand{\OrthogTens}{\TENSTYPESETTYPE{Q}}
\newcommand{\RotTens}{\TENSTYPESETTYPE{R}}
\newcommand{\KronDelta}[1]{\delta\tensopsubscrmath{#1}}

% ---------------------------------------------------------------- basis tensors
\def\WHICHBASESTYLE{\textbf}   % {\mathcal}
\newcommand{\BaseTensP}{\WHICHBASESTYLE{E}\tensopsubscrtext{p}}
\newcommand{\BaseTensZ}{\WHICHBASESTYLE{E}\tensopsubscrtext{z}}
\newcommand{\BaseTensR}{\WHICHBASESTYLE{E}\tensopsubscrtext{r}}
\newcommand{\BaseTensS}{\WHICHBASESTYLE{E}\tensopsubscrtext{s}}
\newcommand{\BaseTensJ}{\WHICHBASESTYLE{E}\tensopsubscrtext{j}}
\newcommand{\BaseTensp}{\BaseTens{p}}
\newcommand{\BaseTensz}{\BaseTens{z}}
\newcommand{\BaseTenss}{\BaseTens{s}}
\newcommand{\BaseTensj}{\BaseTens{j}}
\newcommand{\Er}[1]{\BaseTensR^{#1}}
\newcommand{\Erh}{\Er{\ast}}
\newcommand{\TrialEr}[1]{\Trial{{\Er{#1}}}}
\newcommand{\dEr}{\INCRTYP{\WHICHBASESTYLE{E}}\tensopsubscrtext{r}}
\newcommand{\Ez}{\BaseTensZ}

% ---------------------------------------------------------- fourth order tensors
\newcommand{\AAA}{\mathbb{A}}
\newcommand{\BBB}{\mathbb{B}}
\newcommand{\CCC}{\mathbb{C}}
\newcommand{\FourthOIdent}{{\mathbb{I}}}
\newcommand{\FourthOBaseTens}{\mathbb{E}}
\newcommand{\ESD}{\tenssymdevpart{\FourthOBaseTens}}
\newcommand{\EISO}{\tensisopart{\FourthOBaseTens}}
\newcommand{\ESKEW}{\tensskewpart{\FourthOBaseTens}}
\newcommand{\ProjOp}{\mathbb{P}}
\newcommand{\ProjTens}{\mathbb{P}}
\newcommand{\ArbProjTens}{\mathcal{P}}
\newcommand{\ProjTensCoef}{\Gamma}

% ------------------------------------------------------- arbitrary order tensors
\newcommand{\ArbOTens}{\boldsymbol{\mathcal{A}}}


% * * * * * * * * * * * * * * * * * * * * * * * * * * * * * * * * * * * * * * * %
% * * * * * * * * * * * *  T H E O R E M  M A C R O S   * * * * * * * * * * * * %
% * * * * * * * * * * * * * * * * * * * * * * * * * * * * * * * * * * * * * * * %
% ----------------------------------------------------------------------------- %
% -- Mathematical environment
% ----------------------------------------------------------------------------- %
%\numberwithin{equation}{section}
%\theoremstyle{plain}

% ----------------------------------------------------------------------------- %
%\newtheorem{thm}{Theorem}[section]
%\newtheorem{cor}[thm]{Corollary}
%\newtheorem{lem}[thm]{Lemma}
%\theoremstyle{remark}
%\newtheorem{rem}[thm]{Remark}
%\theoremstyle{definition}
%\newtheorem{dfn}[thm]{Definition}
%
%\newtheorem{theo}{Theorem}[section]
%\newtheorem{prop}[theo]{Proposition}
%\newtheorem{lemm}[theo]{Lemma}
%\newtheorem{coro}[theo]{Corollary}
%% -- definitions & axioms *sand
%\theoremstyle{definition}
%\newtheorem{defi}{Definition}[section]
%\newtheorem{axio}{Axiom}[section]
%% -- remarks & examples *sand
%\theoremstyle{remark}
%\newtheorem{rema}{Remark}[section]
%\newtheorem{exam}{Example}[section]
%% -- algorithms *sand
%\newtheoremstyle{algostyle}% name
%  {}%      Space above, empty = `usual value'
%  {}%      Space below
%  {\sffamily}%         Body font
%  {0pt}%         Indent amount (empty = no indent, \parindent = para indent)
%  {\bfseries}% Thm head font
%  {}%        Punctuation after thm head
%  { }% Space after thm head: \newline = linebreak
%  {\thmname{#1}\thmnumber{ \thesection.#2}{ [#3]}}% Thm head spec
%\theoremstyle{algostyle}
%\newtheorem{algo}{Algorithm}


% ----------------------------------------------------------------------------- %
% ----------------------------------------------------------------------------- %
% ------------ Finite element macros
\newcommand{\Acceleration}{\ddot{\Displacement}}
\newcommand{\Area}{A}
\newcommand{\Displacement}{\boldsymbol{\DisplacementSym}}
\newcommand{\DisplacementSym}{u}
\newcommand{\Force}{\boldsymbol{\ForceSym}}
\newcommand{\ForceExt}{\Force\TSubS{ext}}
\newcommand{\ForceInt}{\Force\TSubS{int}}
\newcommand{\ForceSym}{f}
\newcommand{\MassMatrix}{\boldsymbol{M}}
\newcommand{\Velocity}{\dot{\Displacement}}







% * * * * * * * * * * * * * * * * * * * * * * * * * * * * * * * * * * * * * * * %
% * * * * * * * * * * * * * *  M I S C  M A C R O S   * * * * * * * * * * * * * %
% * * * * * * * * * * * * * * * * * * * * * * * * * * * * * * * * * * * * * * * %
% ------------------------------------------------------------------- misc macros
% --------------------------------------------------------------------- fractions
\newcommand{\OneSixth}{\frac{1}{6}}
\newcommand{\OneFifth}{\frac{1}{5}}
\newcommand{\OneFourth}{\frac{1}{4}}
\newcommand{\OneThird}{\frac{1}{3}}
\newcommand{\OneHalf}{\frac{1}{2}}
\newcommand{\TwoThird}{\frac{2}{3}}
\newcommand{\ThreeFourth}{\frac{3}{4}}
\newcommand{\FourThird}{\frac{4}{3}}
\newcommand{\ThreeHalf}{\frac{3}{2}}
\newcommand{\Toor}[1]{\frac{1}{\sqrt{#1}}}
\newcommand{\OneOver}[1]{\frac{1}{#1}}

% ------------------------------------------------------------------------ arrows
\newcommand{\la}{\leftarrow}
\newcommand{\La}{\Leftarrow}
\newcommand{\lla}{\longleftarrow}
\newcommand{\Lla}{\Longleftarrow}
\newcommand{\lra}{\longrightarrow}
\newcommand{\Lra}{\Longrightarrow}
\newcommand{\ra}{\rightarrow}
\newcommand{\Ra}{\Rightarrow}

% --------------------------------------------- shortcuts for 1st, 2nd, 3rd, etc.
\newcommand{\nst}{^{\text{st}}}
\newcommand{\nnd}{^{\text{nd}}}
\newcommand{\nrd}{^{\text{rd}}}
\newcommand{\second}{2$^{\text{nd}}$}
\newcommand{\nth}{^{\text{th}}}

% ------------------------------------------------------------------------ colors
\newcommand{\Black}[1]{\textcolor{black}{#1}}
\newcommand{\Blue}[1]{\textcolor{blue}{#1}}
\newcommand{\Brown}[1]{\textcolor{brown}{#1}}
\newcommand{\Green}[1]{\textcolor{green}{#1}}
\definecolor{orange}{rgb}{1,0.5,0}
\newcommand{\Orange}[1]{\textcolor{orange}{#1}}
\definecolor{purple}{rgb}{1,0,1}
\newcommand{\Purple}[1]{\textcolor{purple}{#1}}
\newcommand{\Red}[1]{\textcolor{red}{#1}}
\newcommand{\Yellow}[1]{\textcolor{yellow}{#1}}


% --------------------------------------------------------------- misc. shortcuts

% ----------------------------------------------------------------------------- a
\newcommand{\Axial}{\TSubS{axial}}
\newcommand{\AccousticTens}{\TENSTYPESETTYPE{A}}
\newcommand{\GenAccousticTens}{\mechopgenquant{\AccousticTens}}
% ----------------------------------------------------------------------------- b
\newcommand{\BoltzConst}{K\TSubS{B}}
% ----------------------------------------------------------------------------- c
\newcommand{\Comment}[1]{{\Red{\sc#1}}}
\newcommand{\Center}[1]{\begin{center}#1\end{center}}
\newcommand{\cf}{cf.~}
% ----------------------------------------------------------------------------- d
\newcommand{\dd}{d}
\newcommand{\Dashes}{\ensuremath{{\SSCRSIZE-\,-\,-\,-\,-\,-}}}
\newcommand{\DashDot}{\ensuremath{{\scriptstyle-\cdot-\cdot-\cdot-\cdot-}}}
\newcommand{\DashDotDot}{\ensuremath{{\scriptstyle-\cdot\cdot-\cdot\cdot-\cdot\cdot-}}}
\newcommand{\DotDot}{\ensuremath{{\scriptstyle\cdots\cdots\cdots\cdots}}}
\newcommand{\DashDash}{---\,--\,---\,--}
\newcommand{\Line}{---------}
\newcommand{\DiffMeas}[1]{\phi\MSubS{#1}}

% ----------------------------------------------------------------------------- e
%\renewcommand{\eqref}[1]{Eq.~(\ref{#1})}
\newcommand{\eqrefs}[2]{Eqs.~(\ref{eq:#1})--(\ref{eq:#2})}
% ----------------------------------------------------------------------------- h
\newcommand{\HEL}{^{\SSSCRSIZE{\mathrm{HEL}}}}
\newcommand{\Heavyside}[1]{H\left[#1\right]}
% ----------------------------------------------------------------------------- i
%\newcommand{\I}{{_{\SSSCRSIZE{0}}}}
\newcommand{\I}{\MSubS{0}}
% ----------------------------------------------------------------------------- k
\newcommand{\KeyResult}[1]{\textbf{#1}}
% ----------------------------------------------------------------------------- l
\newcommand{\Lateral}{\TSubS{lateral}}
% ----------------------------------------------------------------------------- m
\newcommand{\MathSubScr}[1]{_{\SSCRSIZE{#1}}}
\newcommand{\MathSupScr}[1]{^{\SSCRSIZE{#1}}}
\newcommand{\MSubS}[1]{_{\SSCRSIZE{#1}}}
\newcommand{\MSupS}[1]{^{\SSCRSIZE{#1}}}
% ----------------------------------------------------------------------------- n
\newcommand{\New}{^{\SSCRSIZE\text{new}}}
% ----------------------------------------------------------------------------- o
\newcommand{\Old}{^{\SSCRSIZE\text{old}}}
% ----------------------------------------------------------------------------- p
\newcommand{\pp}{\partial}
\newcommand{\PlusOnTimes}{{{\hbox{$\times$}}%
                    {\kern -.78em\hbox{$+$}}}}
% ----------------------------------------------------------------------------- s
\newcommand{\Star}{^{\ast}}
\newcommand{\Sq}{^{2}}
% ----------------------------------------------------------------------------- t
\newcommand{\TextSubScr}[1]{_{\SSCRSIZE\text{#1}}}
\newcommand{\TextSupScr}[1]{^{\SSCRSIZE\text{#1}}}
\newcommand{\TSubS}[1]{_{\SSCRSIZE\text{#1}}}
\newcommand{\TSupS}[1]{^{\SSCRSIZE\text{#1}}}
% ----------------------------------------------------------------------------- v
% ----------------------------------------------------------------------------- y
% ----------------------------------------------------------------------------- z

% ------------------------------------------------------------------- parenthesis
\newcommand{\LP}{\left(}
\newcommand{\RP}{\right(}


\newcommand{\payette}{{\sc{Payette}}}
\newcommand{\runPayette}{{\texttt{runPayette}}}
\newcommand{\testPayette}{{\texttt{testPayette}}}
\newcommand{\payettedir}{\url{$PAYETTE_HOME}}
\newcommand{\toolset}{\url{$PAYETTE_HOME/Toolset}}
\newcommand{\termcommand}[1]{\indent\hspace{.2in}\texttt{\% #1}\newline}
\newcommand{\levelo}{\indent\hspace{.2in}}
\newcommand{\levelt}{\indent\hspace{.4in}}
\newcommand{\levelth}{\indent\hspace{.6in}}
\newcommand{\simblock}{\Blue{\texttt{simulation}}}
\newcommand{\bndryblock}{\Red{\texttt{boundary}}}
\newcommand{\mtlblock}{\Green{\texttt{material}}}
\newcommand{\legsblock}{\Orange{\texttt{legs}}}
\newcommand{\mathpltblock}{\Purple{\texttt{mathplot}}}
\newcommand{\summhtml}{\texttt{\$PWD/TestResults.\$OSTYPE/summary.html}}
\newcommand{\ratfac}{\texttt{ratfac}}
\newcommand{\shparam}{\texttt{kappa}}
\newcommand{\emit}{\texttt{emit}}
\newcommand{\ampl}{\texttt{ampl}}
\newcommand{\nprints}{\texttt{nprints}}
\newcommand{\screenout}{\texttt{screenout}}
\newcommand{\nsteps}{\texttt{nsteps}}
\newcommand{\tim}{\texttt{time}}
\newcommand{\legno}{\texttt{legno}}
\newcommand{\ltyp}{\texttt{ltyp}}

% python commands
\newcommand{\str}{\texttt{str}}
\newcommand{\none}{\texttt{None}}
\newcommand{\lst}{\texttt{list}}
\newcommand{\dict}{\texttt{dict}}
\newcommand{\true}{\texttt{True}}
\newcommand{\false}{\texttt{False}}

\title{\huge{\payette{}:} \newline
  \huge{An Object Oriented Material Model Driver} \newline
  \Large{Developers and Users Guide}}
\author{Tim Fuller and Scot Swan}

\begin{document}
\maketitle

% ----------------------------------------------------------------------------- %
\section{Introduction}
\payette{} is an object oriented material model driver written in python designed
for rapid development and testing of material models.

% --------------------------------------- %
\subsection{Why a Single Element Driver?}
Due to their complexity, it is often over kill to use a finite element code for
constitutive model development. In addition, features such as artificial
viscosity can mask the actual material response from constitutive model
development. Single element drivers allow the constituive model developer to
concentrate on model development and not the finite element response. Other
advantages of the \payette{} (or, more generally, of any stand-alone
constitutive model driver) are

\begin{itemize}
  \item \payette{} is a very small, special purpose, code.  Thus, maintaining and
  adding new features to \payette{} is very easy.

  \item Simulations are not affected by irrelevant artifacts such as artificial
  viscosity or uncertainty in the handling of boundary conditions.

  \item It is straightforward to produce supplemental output for deep analysis of
  the results that would otherwise constitute an unnecessary overhead in a finite
  element code.

  \item Specific material benchmarks may be developed and automatically run
  quickly any time the model is changed.

  \item Specific features of a material model may be exercised easily by the
  model developer by prescribing strains, strain rates, stresses, stress rates,
  and deformation gradients as functions of time.
\end{itemize}

% ----------------------- %
\subsection{Why Python?}
Python is an interpreted, high level object oriented language. It allows for
writing programs rapidly and, because it is an interpreted language, does not
require a compiling step. While this might make programs written in python slower
than those written in a compiled language, modern packages and computers make the
speed up difference between python and a compiled language for single element
problems almost insignificant.

For numeric computations, the numpy and scipy (\url{www.scipy.org}) modules allow
programs written in Python to leverage a large set of numerical routines provided
by the LAPACK, BLASPACK, EIGPACK, etc. Python's APIs also allow for calling
subroutines written in C or Fortran (in addition to a number of other languages),
a prerequisite for model development as most legacy material models are written
in Fortran. In fact, most modern material models are still written in Fortran to
this day.

Python's object oriented nature allows for rapid installation of new material
models.

% -------------------------------- %
\subsection{Historical Background}
\payette{} is an outgrowth of Tom Pucick's \texttt{MMD} and Rebecca Brannon's
\texttt{MED} drivers.  Both these other drivers are written in Fortran.

% ------------------------------ %
\subsection{Simulation Approach}
\payette{} exercises a material model directly by ``driving'' it through user
specified mechanical and electrical inputs.

% ---------------------------- %
\subsubsection{Supported Drivers}
\paragraph{Mechanical}
\begin{itemize}
  \item Direct
  \begin{itemize}
    \item Strain rate
    \item Strain
    \item Deformation gradient
    \item Velocity
    \item Displacement
  \end{itemize}
  \item Inverse
  \begin{itemize}
    \item Stress
    \item Stress rate
  \end{itemize}
\end{itemize}

\paragraph{Electrical}
\begin{itemize}
  \item Direct
  \begin{itemize}
    \item Electric field
  \end{itemize}
\end{itemize}

% ----------------------------------------------------------------------------- %
\section{\payette{} System Requirements}
\payette{} requires Python 2.6 or later with the scipy, numpy, and matplotlib
modules installed.

% --------------------------------------------- %
\subsection{Installing Required Python Modules}
In the sections below, instructions are given for installing the required python
modules via system level package management programs.  Of course, all modules can
be installed from source, but having a package manager handle all of the build
dependencies is infinitely easier in most cases.

% --------------------------------- %
\subsubsection{Installing on Macintosh}
The Apple operating system does not provide a native package manager, but several
third party package managers are available.  The two most common are Fink
(\url{www.finkproject.org}) and MacPorts (\url{www.macports.org}).  The following
instructions apply to MacPorts.

If you do not have MacPorts installed on your system, first install the optional
Developer Tools from the Mac OSX installation DVD that came with your computer.
If you do not have a copy, the developer tools can be downloaded directly from
Apple (\url{https://developer.apple.com/technologies/tools}), but it is a very
large download ($\sim$3 GB).  After installing the Developer Tools, download the
MacPorts binary installer (\url{http://www.macports.org/install.php}) and
install.

Once MacPorts is installed, simply execute the following command from the
terminal (\url{/Applications/Utilities/Terminal.app}): \newline

\termcommand{sudo port install pyXx-scipy pyXx-matplotlib}

\noindent where \texttt{X} and \texttt{x} are the major and minor python release
numbers, respectively.

% ----------------------------- %
\subsubsection{Installing on Linux}
Most Linux distributions come with a native package manager installed that will
fetch and install the required python modules.  On Ubuntu/Debian, simply execute
in a terminal window:\newline

\termcommand{sudo apt-get python-numpy python-scipy python-matplotlib}

% ----------------------------------------------------------------------------- %
\section{Getting \payette{}}
\label{sec:getting}
The source code for \payette{} is distributed via svn from
\url{svn://lenny.eng.utah.edu/csm/local/svn/Payette/}.  The repository is access
limited.  Contact Tim Fuller (\url{tjfulle@sandia.gov}) or Scot Swan
(\url{mswan@sandia.gov}) to gain access to the repository.  Once access is
granted, execute \newline

\termcommand{svn co svn://lenny.eng.utah.edu/csm/local/svn/Payette/}

\noindent at a command prompt.  Once executed, the following directories will be
checked out on to your local machine

\begin{verbatim}
Payette/
  branches/
  tags/
  trunk/
    Aux/
    Documents/
    Source/
    Tests/
    Toolset/
\end{verbatim}

% ----------------------------------------------------------------------------- %
\section{Building \payette{}}

% -------------------------------------- %
\subsection{Setting Up Your Environment}
\payette{} requires the following additions to your standard environment
\begin{itemize}
  \item \url{PAYETTE_HOME} environment variable that points to
  \url{/path/to/Payette/trunk/}.  \newline
  \indent\hspace{.1in}bash: \texttt{export
    PAYETTE\_HOME=/path/to/Payette/trunk}\\
  \indent\hspace{.1in}csh: \texttt{setenv PAYETTE\_HOME /path/to/Payette/trunk}
  \item Append \url{$PAYETTE_HOME/Toolset} to your \url{$PATH} \\
  \indent\hspace{.1in}bash: \texttt{export PATH =
    \$PATH:\$PAYETTE\_HOME/Toolset}\\
  \indent\hspace{.1in}csh: \texttt{setenv PATH \$PATH:\$PAYETTE\_HOME/Toolset}

  \item If you are authorized to work on the Kayenta material model, set the
  environment variable \url{PAYETTE_KAYENTA} to point to the Kayenta
  \url{trunk/src} directory.
\end{itemize}

% --------------------------------------------------- %
\subsection{Building the \payette{} Executable Scripts}
If all of the previous steps were successfully completed, building \payette{} is
completed through the \texttt{buildPayette} script in the
\url{$PAYETTE_HOME/Toolset} directory.  To build, execute \\

\termcommand{cd \$PAYETTE\_HOME/Toolset}
\termcommand{python buildPayette}

\noindent at a command prompt, making sure that the python interpreter used to
build Payette is the same interpreter that you previously installed numPy and
sciPy support. The \url{buildPayette} script will build two additional scripts in
\url{$PAYETTE_HOME/Toolset}: \runPayette{} and \testPayette{}, both of
which are described later in this document.

\url{buildPayette} has additional options/capabilities that can be seen by
executing: \\

\termcommand{python buildPayette -h}

% ----------------------------------------------------------------------------- %
\section{Running \payette{}}
Interacting with \payette{} is done through the \runPayette{} script and
properly formatted input files.  The basic usage of \runPayette{} is \\

\termcommand{runPayette input\_file}

For a complete list of options for \runPayette{} execute \\

\termcommand{runPayette -h}

% ----------------------------------------------------------------------------- %
\subsection{Simulation Output}
For a simulation titled ``\url{simnam}'', the following output is created by
\runPayette{}:
\begin{description}
  \item[\url{simnam.log}] ascii log file
  \item[\url{simnam.out}] ascii space delimited output file
  \item[\url{simnam.math1}] ascii Mathematica auxiliary postprocessing file
  \item[\url{simnam.math2}] ascii Mathematica auxiliary postprocessing file
  \item[\url{simnam.prf}] binary restart file
  \item[\url{simnam.props}] ascii list of checked material parameters
\end{description}

% ----------------------------------------------------------------------------- %
\section{\payette{} Input File Formatting}
Input files are comprised of several ``blocks'' of instruction for \runPayette{}.
A block is a group of instructions contained in a \texttt{begin <block> [block
  name] ... end <block>} pair:

\begin{verbatim}
begin <block> [block name]
        .
        .
        .
end <block>
\end{verbatim}

% -------------------------------------------------- %
\paragraph{A Note on Case, Spacing, and Comment Characters}
For the most part, the case of the text in the input file does not matter, nor
does the indentation on each line. Indentation of input blocks is used only for
clarity in this document. \runPayette{} supports \# and \$ as native comment
characters and the user can pass the optional \url{--cchar=userchar} to specify
any character ``\url{userchar}'' to be used as a comment character.

% ------------------------- %
\subsection{Required Blocks}
The blocks required by \runPayette{} are

\levelo\Blue{\texttt{begin simulation <title>}}\\
\levelt\Green{\texttt{begin material}}\\
\levelth\Green{\texttt{[material options]}}\\
\levelt\Green{\texttt{end material}}\\
\levelt\Red{\texttt{begin boundary}}\\
\levelth\Red{\texttt{[boundary options]}}\\
\levelt\Red{\texttt{end boundary}}\\
\levelo\Blue{\texttt{end simulation}}\\

The ordering of the blocks within the \simblock{} block does not
matter. However, all blocks for a given simulation must be nested in the
\simblock{} block. Details of the required content of each block
follows.

% --------------------------- %
\subsubsection{\simblock{} Block}
Each input file must have a \simblock{} block with title for that simulation. The
title of the simulation will serve as the basename for all simulation output,
with spaces replaced with underscores.

% ------------------------------------- %
\paragraph{Number of Simulations per Input File}
\runPayette{} supports an arbitrary number of \simblock{} blocks in a single
input file.


% --------------------------------------------------------- %
\subsubsection{\mtlblock{} Block}
In the \mtlblock{} block, the constitutive model and material parameters are
defined. A \mtlblock{} block would look like

\levelo\Green{\texttt{begin material}}\\
\levelt\Green{\texttt{constitutive model <model>}}\\
\levelt\Green{\texttt{parameter 1 <value>}}\\
\levelt\Green{\texttt{parameter 2 <value>}}\\
\levelt\Green{\texttt{parameter n <value>}}\\
\levelo\Green{\texttt{end material}}\\

An important philosophical point, parameters are associated with the
material, and not the constitutive model.  This allows different materials to be
exercised by different constitutive models without changing parameters.

An example \mtlblock{} input block for an elastic material would look like

\levelo\Green{\texttt{begin material}}\\
\levelt\Green{\texttt{constitutive model elastic}}\\
\levelt\Green{\texttt{bkmod 130.e9}}\\
\levelt\Green{\texttt{shmod 57.e9}}\\
\levelo\Green{\texttt{end material}}

% Another option is not specifying each parameter seperately but through a
% parameters file, do this by

% \indent\hspace{1.0in}\Green{\texttt{begin material}}\\
% \indent\hspace{1.2in}\Green{\texttt{constitutive model <model>}}\\
% \indent\hspace{1.2in}\Green{\texttt{material <material\_name>}}\\
% \indent\hspace{1.0in}\Green{\texttt{end material}}\\

% This directive tells \payette{} to look for the file \texttt{material\_name.py} first
% in the current directory and then in \url{$PAYETTEDIR/Aux/Materials/Database}.
% The \texttt{material\_name.py} contains a single python dictionary
% \texttt{parameters} with \texttt{'parameter name':value} key:value pairs. The
% parameter file for 6061 T6 aluminum
% (\url{$PAYETTEDIR/Aux/Materials/Database/al_6061.py}) looks like

% \begin{verbatim}
% ``` Material properties file
%     MATERIAL: 6061 T6 Aluminum
%     SOURCE:   MatWeb (www.matweb.com)
%     UNITS:    Kg/m/s/K
% '''

% parameters = {`initial density':2700,
%               `brinell hardness':95,
%               `knoop hardness':120,
%               `rockwell a hardness':40,
%               `rockwell b hardness':60,
%               `vickers hardness':107,
%               `tensile strength':276e6,
%               `ultimate tensile strength':310e6,
%               `shear strength':207e6,
%               `ultimate bearing strength':607e6,
%               `bearing strength':386e6,
%                         .
%                         .
%                         .
%               `bulk wave speed':5240}
% \end{verbatim}

% ------------------------------ %
\subsubsection{\bndryblock{} Block}
\label{sec:boundary}
In the \bndryblock{} block, the boundary conditions for the simulation are
defined. The \bndryblock{} block has the following format, with the default
values being shown.

\levelo\Red{\texttt{begin boundary}}\\
\levelt\Red{\texttt{kappa = 0.}}\\
\levelt\Red{\texttt{ampl = 1.}}\\
\levelt\Red{\texttt{ratfac = 1.}}\\
\levelt\Red{\texttt{tstar = 1.}}\\
\levelt\Red{\texttt{sstar = 1.}}\\
\levelt\Red{\texttt{estar = 1.}}\\
\levelt\Red{\texttt{fstar = 1.}}\\
\levelt\Red{\texttt{dstar = 1.}}\\
\levelt\Red{\texttt{efstar = 1.}}\\
\levelt\Red{\texttt{stepstar = 1.}}\\
\levelt\Red{\texttt{emit = all \{all,sparce\}}}\\
\levelt\Red{\texttt{screenout = false \{true,false\}}}\\
\levelt\Red{\texttt{nprints = 0 \{0-nsteps\}}}\\
\levelt\Orange{\texttt{begin legs}}\\
\levelth\Orange{\texttt{\# leg no,  time, nsteps, ltyp,  c[ij]}}\\
\levelt\Orange{\texttt{end legs}}\\
\levelo\Red{\texttt{end boundary}}\\

% ----------------------------------------- %
\paragraph{Keyword ``[t,s,e,f,d,ef,step]star''}
Multiplier on all components of time, stress, strain, deformation gradient,
strain rate, displacement, electric field, and number of steps, respectively. All
values of the previously listed quantities are defined in each leg will be
multiplied by this factor. As an example, if the simulation times are given in
microseconds, ``\url{tstar}'' could be set to ``1.0e-6'' and the times given in
integer microsecond values.

% --------------------------- %
\paragraph{Keyword ``\emit{}''}
Write all data (\emit{} = \url{all}) or data from only 10 timesteps (\emit{} =
\url{sparse}) to the output file.

% --------------------------- %
\paragraph{Keyword ``\screenout{}''}
Print out all timestep information to the console.

% --------------------------- %
\paragraph{Keyword ``\nprints{}''}
Total number of writes to the output file during the simulation.

% --------------------------- %
\paragraph{Keyword ``\ampl{}''}
Multiplier on all leg inputs.  \ampl{} may be used to increase or decrease the peak
values of the given inputs without changing the rates of those inputs.

% --------------------------- %
\paragraph{Keyword ``\ratfac{}''}
Multiplier on strain and stress rates - effectively achieved by dividing each
time by \ratfac{}.

% --------------------------- %
\paragraph{Keyword ``\shparam{}''}
The keyword ``\shparam{}'' is only used/defined for the purposes of strain or strain
rate control. It refers to the coefficient used in the Seth-Hill general strain
definition. For your convenience, the formula used is
%
\begin{equation}
  [\varepsilon] = \frac{1}{\kappa}\left( [U]^\kappa - [I]\right)
\end{equation}
%
Where $\kappa$ is the keyword ``\shparam{}'', $[\varepsilon]$ is the strain tensor,
$[U]$ is the right Cauchy stretch tensor, and $[I]$ is the identity tensor. Below
in Table \ref{table:sethhill} is a table of common values of \shparam{} and the
associated common names for each (there is some ambiguity in the names).

\begin{table}[!htb]
  \centering
  \begin{tabular}{|c|c|}
    \hline
    \multicolumn{2}{|c|}{Common Strain Names} \\
    \hline
    $\kappa$ & Name(s) \\ \hline
    $-2$ & Green \\
    $-1$ & True, Cauchy \\
    $0$ & Logarithmic, Hencky, True \\
    $1$ & Engineering, Swainger \\
    $2$ & Lagrange, Almansi\\
    \hline
  \end{tabular}
  \caption{Sampling of names encountered in the literature for common strain
    measures in the Seth-Hill family of strains for integer values of $\kappa$.
    \label{table:sethhill}}
\end{table}

\subsubsection{\legsblock{} Block}
The \legsblock{} block defines the material states that will be applied to the
single element throughout the simulation. In the most general case, each leg will
be defined as follows:

\levelo\Orange{\texttt{begin legs}}\\
\levelt\Orange{\texttt{leg no,  time, nsteps, ltyp,  c[ij]}}\\
\levelo\Orange{\texttt{end legs}}

The ``\legno{}'' or leg number is always a strictly monotonically increasing
integer, starting from zero for the first leg. There can be an arbitrary number
of legs defined.

The value ``\tim{}'' defines at what time in the simulation the material state will
be as it is defined in that leg. Generally the first leg (leg zero) will have a
time equal to 0 (seconds, microseconds, etc.). The values of time must increase
strictly monotonically with leg number.

The value of ``\nsteps{}'' is an integer that defines the number of steps the
simulation will take to get from the previous leg to the current leg. Currently,
it is not possible to explicitly define a timestep for the single element tests.
However, by setting the time increment and value of ``\nsteps{}'' you can set the
timestep size for that leg.

The value of ``\ltyp{}'' or leg type is a little more involved. This keyword
frequently has more features added to it, most of which are experimental.
However, in this document only the most basic and stable options will be
addressed.

The basic form of the leg type is a string that specifies the material state for
specific components by setting each character to one of the following:

\begin{itemize}
  \setlength{\itemsep}{0pt}
  \setlength{\parskip}{0pt}
  \setlength{\parsep}{0pt}
  \item[\textbf{1}] strain rate control (mech,6)
  \item[\textbf{2}] strain control (mech,6)
  \item[\textbf{3}] stress rate control (mech,6)
  \item[\textbf{4}] stress control (mech,6)
  \item[\textbf{5}] deformation gradient control (mech,9)
  \item[\textbf{6}] electric field (elec,3)
  \item[\textbf{8}] displacement (mech,3)
\end{itemize}

There are two types of control here: mechanics control (mech) and electric field
control (elec). Because these two types are seperated and handled individually
after parsing, the mechanics and electric control characters can be mixed
together without change in behavior. The integer given in the parentheses in the
above list represent the maximum number of components that may be defined. The
user must define at least three mechanics options for any given simulation. If no
electric field options are given, they default to zero.

Once the mechanics control characters are gathered, they are checked for
compatibility. Specifically, the rules are as follows:

\begin{itemize}
  \item Deformation gradient control (5) cannot be mixed with other mechanics
  control options and all 9 components must be defined.
  \item Displacement control (8) cannot be mixed with other mechanics control
  options.
\end{itemize}

Then, the component values ``c[ij]'' are read in and are assigned values based on
the leg type string. The general case for ``c[ij]'' user input components (which
is actually a 1D array) correlating to tensor components is
%
\begin{equation}
  [C] = \left[ \begin{array}{ccc} C_{1} & C_{4} & C_{6} \\ C_{7} & C_{2} & C_{5}\\ C_{9} & C_{8} & C_{3} \end{array}\right]
\end{equation}
%
Payette simply follows this pattern for assigning variables. However, at least the
first three must be defined (the $x$, $y$, and $z$ components). If any variables
are given beyond this, it fills in the matrix in that order up to the maximum
number of components.

\paragraph{Example 1: Deformation Gradient, Uniaxial Extension}
This example extends the material in the $x$-direction.

\levelo\Orange{\texttt{begin legs}}\\
\levelt\Orange{\texttt{0, 0.0, 0, 555555555, 1.0, 1., 1., 0., 0., 0., 0., 0., 0.}}\\
\levelt\Orange{\texttt{1, 1.0, 0, 555555555, 1.1, 1., 1., 0., 0., 0., 0., 0., 0.}}\\
\levelo\Orange{\texttt{end legs}}

\paragraph{Example 2: Strain Control, Uniaxial Extension}
All of the following produce equivalent behavior (uniaxial strain extension).
Remember to set ``\shparam{}'' to the desired value.

\levelo\Orange{\texttt{begin legs}}\\
\levelt\Orange{\texttt{0, 0.0, 0, 222, 0.0, 0., 0.}}\\
\levelt\Orange{\texttt{1, 1.0, 0, 222, 0.1, 0., 0.}}\\
\levelo\Orange{\texttt{end legs}}

\noindent or

\levelo\Orange{\texttt{begin legs}}\\
\levelt\Orange{\texttt{0, 0.0, 0, 2222, 0.0, 0., 0., 0.}}\\
\levelt\Orange{\texttt{1, 1.0, 0, 2222, 0.1, 0., 0., 0.}}\\
\levelo\Orange{\texttt{end legs}}

\noindent or

\indent\hspace{0.2in}\Orange{\texttt{begin legs}}\\
\indent\hspace{0.4in}\Orange{\texttt{0, 0.0, 0, 222222, 0.0, 0., 0., 0., 0., 0.}}\\
\indent\hspace{0.4in}\Orange{\texttt{1, 1.0, 0, 222222, 0.1, 0., 0., 0., 0., 0.}}\\
\indent\hspace{0.2in}\Orange{\texttt{end legs}}\\

\paragraph{Example 3: Stress Control, Uniaxial Tension}
Stress control can be defined in much the same way as strain control (see example
2). Assuming that the material parameters are defined in MKS, stress will be
defined in terms of Pa.

\levelo\Orange{\texttt{begin legs}}\\
\levelt\Orange{\texttt{0, 0.0, 0, 444444, 0.0e9, 0., 0., 0., 0., 0.}}\\
\levelt\Orange{\texttt{1, 1.0, 0, 444444, 1.0e9, 0., 0., 0., 0., 0.}}\\
\levelo\Orange{\texttt{end legs}}\\

\paragraph{Example 4: Mixed Control}
This example could be used to reproduce the stress state found in the standard
tension test when the strain is known. Specifically, we prescribe the strain
value in the $x$-direction while prescribing the lateral stresses to be held at
zero stress.

\levelo\Orange{\texttt{begin legs}}\\
\levelt\Orange{\texttt{0, 0.0, 0, 244, 0.00, 0., 0.}}\\
\levelt\Orange{\texttt{1, 1.0, 0, 244, 0.01, 0., 0.}}\\
\levelo\Orange{\texttt{end legs}}\\


\subsubsection{Notes on Stress Control}
As prescribed stress is essentially an inverse problem, prescribed stress
simulations take much longer than prescribed strain simulations. This is due to
the need of using optimization routines to determine the correct strain increment
to input to the material model.

The general objective function used in the optimization routine is simply an L2
norm of the difference of the prescribed stress components with the output stress
components. This is of particular importance when a stress is prescribed outside
of the limit surface for a material and the material stress simply cannot reach
the prescribed stress. For example, if the user desired uniaxial stress but
prescribed the stress to some value beyond the limit surface then the stress
state might develop non-zero values in components that could reasonably be
expected to stay zero.

To alleviate this problem, Payette has a command line argument ``--proportional''
that enforces proportional loading for prescribed stress problems. This changes
the objective function to optimize for a stress state that is proportional to the
prescribed stress by some factor and that is as close as possible to the
prescribed stress. Specifically, it decomposes the output stress tensor into
``proportional'' and ``perpendicular'' tensors with the proportional stress
tensor begin a multiple of the prescribed stress tensor and the perpendicular
tensor being the balance. The function takes the square of the L2 norm of the
perpendicular tensor and the L2 norm of the difference of the proportional and
prescribed tensors. In this way Payette weights the stress state such that being
proportional is more important than being closer. For this objective function
using the same simulation as described for the default objective function, we
would maintain uniaxial stress even when the limit surface is attained.

As a warning, stress control (especially proportional loading) can introduce
excessive noise when stress states are prescribed outside of the yield surface.
However, it can also perform flawlessly. Be aware of the limits of the objective
functions when choosing.

% ------------------------- %
\subsection{Optional Blocks}
The following blocks are optional: \mathpltblock{}.

% ------------------------------- %
\subsubsection{\mathpltblock{} Block}
The \mathpltblock{} block is used to specify optional plotting information
written to the \url{simnam.math[1,2]} output files for use in Rebecca Brannon's
Mathematica post processing files. The basic syntax is

\levelo\Purple{\texttt{begin mathplot}} \\
\levelt\Purple{var1 var2 var3} \\
\levelt\Purple{var4, var5, var6} \\
\levelt\Purple{var7; ...; varn} \\
\levelo\Purple{\texttt{end mathplot}}

\noindent where var? are \payette{} and material model variables.  A complete
list of plotable variables is listed in each simulation's log file.  Each line in
the \mathpltblock{} block can contain an arbitrary number of space, comma, or
semi-colon delimited variables.

% -------------------------------- %
\subsection{Inserting External Files}
External files containing formatted \payette{} input can be included anywhere in
the input file through the \texttt{include} and \texttt{insert} directives.  For
example, material parameters can be kept in a separate parameter file and
inserted in to an input file by

\levelo\Green{\texttt{begin material}}\\
\levelt\Green{\texttt{constitutive model kayenta}}\\
\levelt\Green{\texttt{insert salem\_limestone.dat}}\\
\levelo\Green{\texttt{end material}}


\noindent When \payette{} encounters an [\texttt{insert,include}] directive, it
looks for the inserted file by it's absolute path, in the current directory, and
in the \url{$PAYETTE_HOME/Aux/MaterialsDatabase}, in that order.

\clearpage
% ----------------------------- %
\subsection{Complete Input File}
Below is an input file, any keywords not given take the default values shown in
Section \label{sec:boundary}.

\begin{verbatim}
begin simulation elastic unistrain cycle
  begin material
    constitutive model elastic
    shmod 53.e9
    bkmod 135.e9
  end material
  begin boundary
    kappa = 0.
    tfac = 1.
    amplitude = 1
    begin legs
#     l  t   n     ltyp       c[ij]...
      0, 0., 0.,   222222,    0., 0., 0., 0., 0., 0.
      1, 1., 100,  222222,    .1, 0., 0., 0., 0., 0.
      2, 2., 100,  222222,    0., 0., 0., 0., 0., 0.
      3, 3., 100,  444444,    20.566e9, 9.966e9, 9.966e9, 0., 0., 0.
      4, 4., 100,  444444,    0. ,0. ,0. ,0. ,0. ,0.
      5, 5., 100,  111111,    0.1 ,0. ,0. ,0. ,0. ,0.
      6, 6., 100,  111111,   -0.1 ,0. ,0. ,0. ,0. ,0.
      7, 7., 100,  333333,    20.566e9, 9.966e9, 9.966e9, 0., 0., 0.
      8, 8., 100,  333333,   -20.566e9, -9.966e9, -9.966e9, 0., 0., 0.
      9, 9., 100,  555555555, 1.1052, 1., 1., 0., 0., 0., 0., 0., 0.
     10, 10.,100,  555555555, 1., 1., 1., 0., 0., 0., 0., 0., 0.
    end legs
  end boundary
  begin mathplot
    sig11 sig22 sig33
    eps11 eps22 eps33
  end mathplot
end simulation
\end{verbatim}

\clearpage
% ----------------------------------------------------------------------------- %
\section{\testPayette{}}
\testPayette{} is what makes \payette{} an invaluable tool to material model
development. Simplistically, \testPayette{} is a tool that exercises
\runPayette{} on previously setup simulations and compares the results against a
``gold'' result, allowing model developers to instantly see if changes to a
material model affect its results.

% ----------------------------- %
\subsection{Basic Usage of \testPayette{}}
From a command prompt execute\\

\termcommand{testPayette}

\noindent \testPayette{} will then scan \url{$PAYETTE_HOME/Tests} for tests.
After creating a lists of tests to run, \runPayette{} will create
\url{$PWD/TestResults.$OSTYPE} directory and subdirectories in which the tests
will be run. The subdirectories created in \url{$PWD/TestResults.$OSTYPE/} mirror
the subdirectories of \url{$PAYETTE_HOME/Tests}. After creating the
\url{$PWD/TestResults.$OSTYPE/} directory tree, it will then move in to the
appropriate subdirectory to run and analyze each individual test. For example, if
the \url{Tests/Materials/Elastic/elastic-unistrain.py} test is run,
\testPayette{} will create a
\url{$PWD/TestResults.$OSTYPE/Materials/Elastic/elastic-unistrain/} directory,
move to it, and run the \url{elastic-unistrain.py} test in it.

% ------------------------------- %
\subsection{Filtering Tests to Run}
The tests run can be filtered by keyword with the \url{-k} option. For example,
to run only the ``fast'' ``kayenta'' material tests, execute \\

\termcommand{testPayette -k kayenta -k fast} \\

\noindent Or, you can run specific tests with the \url{-t} option.  For example,
to run only the \url{elastic-unistrain.py} test, execute\\

\termcommand{testPayette -t elastic-unistrain} \\

% --------------------------------- %
\subsection{Multiprocessor Support}
By default, \testPayette{} will run all collected tests serially on a single
processor, but supports running tests on multiple processors with the \texttt{-j
  nproc} option, where \texttt{nproc} is the number of processors.  For example,
to run all of the kayenta tests on 8 processors, execute \\

\termcommand{testPayette -k kayenta -j8}

\noindent Be warned, however, that when run on multiple processors, you cannot
kill \testPayette{} by \texttt{ctrl-c}, but must put the job in the background
and kill it with \texttt{kill -9 pid}, where \texttt{pid} is the process id. Any
help from a python multiprocessor expert on getting around this limitation would
be greatly appreciated!

% ------------------------------------- %
\subsection{Other \testPayette{} Options}
Execute \\

\termcommand{testPayette -h}

\noindent for a complete list of options.

% ------------------------------------------------------------- %
\subsection{\summhtml{}}
\testPayette{} gives a complete summary of test results in \summhtml{} viewable
in any web browser.

% ----------------------------------------------------------------------------- %
\subsection{Mathematca Post Processing}
For the Kayenta material model, \testPayette{} creates Prof. Brannon's
Mathematica post processing file \url{kayenta.nb} (formerly \url{030626.nb}) in
\url{$PWD/TestResults.$OSTYPE/Materials/Kayenta/}, that aids greatly in analyzing
test results for the Kayenta material model.

\subsection{Creating New Tests}
\Comment{not done yet!}


% ----------------------------------------------------------------------------- %
\section{Installing New Materials in \payette{}}
\payette{} was born from the need for an environment in which material
constitutive models could be rapidly developed, tested, deployed, and maintained,
independent of host finite element code implementation. Important prerequisites
in the design of \payette{} were ease of model installation and support for
constitutive routines written in Fortran. This requirement is met by providing a
simple API with which a model developer can install a material in \payette{} as a
new Python class and use \texttt{f2py} to compile Python extension modules from
the material's Fortran source (if applicable). In this section, the \payette{}
API is described through an example elastic material model and instructions for
extending a material model to include Fortran subroutines are given.

% ------------------------------------------- %
\subsection{Material File Naming Conventions}
Each material model must provide its Python class definition in its own file in
the \url{$PAYETTE_HOME/Source/Materials}.  The accepted naming convention for the
material's Python class file is \newline

\url{$PAYETTE_HOME/Source/Materials/Payette_material_name.py} \newline

A material's Fortran source code (if applicable) is stored in its own directory
in the \url{$PAYETTE_HOME/Source/Materials/Fortran} directory. The accepted
naming convention for the material's Fortran directory is \newline

\url{$PAYETTE_HOME/Source/Materials/Fortran/MaterialName} \newline

There is no defined naming convention for Fortran source code within the
materials \url{Fortran/MaterialName} directory.

As an example, suppose you were developing a new material model ``Porous Rock''.
The source files and directories would be named: \newline

\url{$PAYETTE_HOME/Source/Materials/Payette_porous_rock.py} \newline
\url{$PAYETTE_HOME/Source/Materials/Fortran/PorousRock}

% ------------------------------------------------- %
\subsubsection{A Note On Including Fortran Source Code}
The \payette{} project is an open source project, without restrictions on how the
source code is used and/or distributed. However, many material models do have
restrictive access controls and cannot, therefore, include their source files in
\payette. This is the case for many materials currently being developed with
\payette. For these materials, rather than include the source code in the
material's Fortran directory, the Fortran build script used by \payette{} to
build the material's Python extension module is used to direct \payette{} to the
location of the material's source files, elsewhere in the developer's file
system. The contents of the Fortran build script are discussed later in this
document in the section on building Fortran source code in \payette.

% ---------------------------------------------------------------------------- %
\subsection{\texttt{ConstitutiveModel} -- Constitutive Model Base Class}
\payette{} provides a simple interface for interacting with material models
through the Python class structure. Material models are installed as separate
Python classes, derived from the \texttt{ConstitutiveModel} base class. The
\texttt{ConstitutiveModel} base class offers the following methods in its API:

\subsubsection{Data}

\begin{description}
  \item{\texttt{ConstitutiveModel}.\textbf{name}} [type:\str, default:\none]\\
  Constitutive model name, used in parsing user input.

  \item{\texttt{ConstitutiveModel}.\textbf{aliases}} [type:\lst, default:[]]\\
  Constitutive model aliases, used in parsing user input.

  \item{\texttt{ConstitutiveModel}.\textbf{parameter\_table}}
  [type:\dict, default:\{\}] \newline
  Table of parameter names, position in material user input array, and aliases.
  Used in parsing user input. Derived classes must return a parameter\_table of the form\\
  \url{param_table} = \{\\
  ``\url{param_1_name}'':\{``ui pos'':0,``aliases'':[``alias 1'',``alias 2'',...]\},\\
  ``\url{param_2_name}'':\{``ui pos'':1,``aliases'':[``alias 1'',``alias 2'',...],\\
  ... \\
  ``\url{param_n_name}'':\{``ui pos'':n-1,``aliases'':[``alias 1'',``alias 2'',...]\}\\
  \}
  \item{\texttt{ConstitutiveModel}.\textbf{nprop}} \newline
  type:int
  default:0

  \item{\texttt{ConstitutiveModel}.\textbf{ndc}} \newline
  type:int
  default:0

  \item{\texttt{ConstitutiveModel}.\textbf{nsv}} \newline
  type:int
  default:0

  \item{\texttt{ConstitutiveModel}.\textbf{ui0}} \newline
  type:numpy.array\newline
  default:numpy.zeros(self.nprop)

  \item{\texttt{ConstitutiveModel}.\textbf{ui}} \newline
  type:numpy.array\newline
  default:numpy.zeros(self.nprop)

  \item{\texttt{ConstitutiveModel}.\textbf{dc}} \newline
  type:numpy.array\newline
  default:numpy.zeros(self.ndc)

  \item{\texttt{ConstitutiveModel}.\textbf{namea}} \newline
  type:list\newline
  default: [None]*self.nsv

  \item{\texttt{ConstitutiveModel}.\textbf{keya}} \newline
  type:list\newline
  default: [None]*self.nsv

  \item{\texttt{ConstitutiveModel}.\textbf{bulk\_modulus}} \newline
  type:real\newline
  default: 0.

  \item{\texttt{ConstitutiveModel}.\textbf{shear\_modulus}} \newline
  type:real\newline
  default: 0.

  \item{\texttt{ConstitutiveModel}.\textbf{multi\_level\_fail\_model}} \newline
  type:bool\newline
  default:False

  \item{\texttt{ConstitutiveModel}.\textbf{electric\_field\_model}} \newline
  type:bool\newline
  default:False

  \item{\texttt{ConstitutiveModel}.\textbf{cflg\_idx}}
  default: None

  \item{\texttt{ConstitutiveModel}.\textbf{fratio\_idx}} \newline
  None

\end{description}

\subsubsection{Methods}

\begin{verbatim}

    def __init__(self, props):
        pass

    def Initialize_State(self,*args,**kwargs):
        pass

    def Set_Up(self,*args,**kwargs):
        msg = 'Consitutive model must provide Set_Up method'
        reportError(__file__,msg)
        return 1

    def Update_State(self,*args,**kwargs):
        msg = 'Consitutive model must provide updateState method'
        reportError(__file__,msg)
        return 1

    def initialParameters(self):
        return self.ui0

    def checkedParameters(self):
        return self.ui

    def modelParameters(self):
        return self.ui

    def initialJacobian(self):
        return self.J0

    def internalStateVariables(self):
        return self.sv

    def derivedConstants(self):
        return self.dc

    def isvKeys(self):
        return self.keya

    def computeInitialJacobian(self):
\end{verbatim}
\begin{itemize}
  \item \texttt{def} \url{__init__(self,props)}
  \item \texttt{def jacobian(self,dt,d,Fold,Fnew,EF,sig,sv,v,*args,**kwargs)}
  \item \texttt{def checkProperties(self,*args,**kwargs)}
  \item \texttt{def setField(self,*args,**kwargs)}
  \item \texttt{def updateState(self,*args,**kwargs)}
\end{itemize}

Once a material object is instantiated by \payette{}, \payette{} interacts with
that model by calling any of the above listed methods.

\paragraph{\texttt{ConstitutiveModel.\textbf{checkProperties}(self,*args,**kwargs)}}
\paragraph{\texttt{ConstitutiveModel.\textbf{setField}(self,*args,**kwargs)}}
\paragraph{\texttt{ConstitutiveModel.\textbf{updateState}(self,*args,**kwargs)}}

% ------------------------------- %
\subsection{Derived Material Class}
% ------------------------------------ %
\subsubsection{\texttt{attributes} Dictionary}

% ----------------------- %
\subsection{Fortran Code}
% ----------------------- %
\subsubsection{Build Script}

\begin{verbatim}
import sys
import os
import numpy as np

from Source.Payette_utils import *
from Source.Payette_constitutive_model import ConstitutiveModelPrototype

attributes = {"payette material":True,
              "name":"py elastic",
              "aliases":["python elastic","py hooke","python hook"],
              "material type":["mechanical"]
              }
class PyElastic(ConstitutiveModelPrototype):
    """
    CLASS NAME
       PyElastic

    PURPOSE

       Constitutive model for an elastic material in python. When instantiated,
       the Elastic material initializes itself by first checking the user input
       (checkProperties) and then initializing any internal state variables
       (setField). Then, at each timestep, the driver update the Material state
       by calling updateState.

    METHODS
       checkProperties
       setField
       updateState

    AUTHORS
       Tim Fuller, Sandia National Laboratories, tjfulle@sandia.gov
    """

    imported = True
    name = attributes["name"]
    aliases = attributes["aliases"]
    parameter_table = {
    "K":{"ui pos":0,"aliases":["BKMOD"]},
    "G":{"ui pos":1,"aliases":["SHMOD"]}}
    nprop = len(parameter_table.keys())

    def __init__(self,props):
        self.dc = np.zeros(0)
        self.ui0 = np.array(props)
        self.ui = self.checkProperties(props=props)
        (self.nsv,self.namea,self.keya,self.sv,
         self.rdim,self.iadvct,self.itype) = self.setField()
        self.J0 = self.computeInitialJacobian()

    def jacobian(self,dt,d,Fold,Fnew,EF,sig,sv,v,*args,**kwargs):
        return self.J0[[[x] for x in v],v]

    def checkProperties(self,*args,**kwargs):
        K,G = kwargs["props"]

        msg = ""
        if K <= 0.0: msg += "Bulk modulus K must be positive.  "
        if G <= 0.0: msg += "Shear modulus G must be positive"
        if msg: reportError(__file__,msg)

        if 3.*K < 2.*G: msg += "neg Poisson (to avoid warning, set 3*B0>2*G0)"
        if msg: reportWarning(__file__,msg)
        self.bulk_modulus,self.shear_modulus = K,G

        # internally, work with lame parameters
        lam = K - 0.66666666666666666667*G
        return np.array([G,lam])

    def setField(self,*args,**kwargs):
        nsv = 2
        sv = np.array([0.,1.])
        namea,keya = ["Free 01","Free 02"], ["F01","F02"]
        rdim,iadvct,itype = [None]*3
        return nsv,namea,keya,sv,rdim,iadvct,itype

    def updateState(self,*args,**kwargs):
        """
           update the material state based on current state and strain increment
        """
        dt,d,fold,fnew,efield,sigold,svold = args
        de, delta = d*dt, np.array([1.,1.,1.,0.,0.,0.])
        dev = (de[0] + de[1] + de[2])
        twog, lam = 2.*self.ui[0], self.ui[1]
        signew = sigold + twog*de + lam*dev*delta
        return signew, self.sv
\end{verbatim}
\end{document}
